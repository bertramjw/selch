\section{Originalrezepte
}
\subsection{Geselchtes au Peter}
\subsubsection*{Ingredients}
\subsubsection*{Directions}
\begin{enumerate}
\item Liebstoeckl, Majoran, Pfeffer, Thymian Kuemmel in heisses Wasser gegen und 3 Tage stehen lassen.
\item Fleisch in Salz Pfeffer Majoran und Knoblauch einreiben und in Behaeltnisgeben. Mit Brettern beschweren und 2-3 Tage liegen lassen
\item Gewuerzwasser ueber das Fleisch geben und 14 Tage stehen lassen.
\item Fleisch abschwemmen und Fleisch mit Spicknadel und Schnur aufhaengen.
\item 1.5 Tage mit Buchenholz raeuchern.
\end{enumerate}
Anmerkung: Die Lisi macht es auch so, aber mit Salz als einizges Gewuerz
\pagebreak


\subsection{Geselchtes a la Martha}
\subsubsection*{Ingredients}
\subsubsection*{Directions}
\begin{enumerate}
\item ca. 2 dag Pökelsalz auf 1 kg Rohfleisch
\item mit  Kümmel Knoblauch Lorbeerblättern - eine Mischung bereiten, und damit das Fleisch einreiben
\item Die eingeriebenen Fleischstücke in ein enges Plastikwanndl  legen und bei ca. 8Grad 
\item am Besten Fleischstücke auf 2 Lagen zB übereinander legen
\item nach ca. 3-4 Tagen - umschichten des Fleisches (das obrig liegende nach unten - und das unten liegende oben auf legen)
\item nach insgesamt 8-10 Tagen Beiz-Ruhedauer kann dann geselcht werden. 
\item Selchen bei uns in unserer Selch ca. 1 Tag - mit Buchenholz, Wacholderästen, ... aber ist jede Selch anders zum handhaben...
\end{enumerate}
Anmerkung: Sie benutzt aber normales Salz statt Poekelsalz
\pagebreak

