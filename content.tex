
\section{China}

\subsection{Jiao Zi}
%\includegraphics[width=90mm]{img/JiaoZi/IMG_1775.jpg}
\subsubsection*{Ingredients (4 persons)}
Teig:
\begin{itemize}
\item[] 500 g Mehl
\item[] 10 EL Wasser
\item[] Prise Salz
\end{itemize}
Fuellung:
\begin{itemize}
\item[] 500 g Hackfleisch vom Schwein
\item[] 400-500 g Chinakohl
\item[] 1 Bund chinesischer Schnittlauch (oder Fruehlingszwiebel)
\item[] 2 Zehen Knoblauch
\item[] Ingwer gehackt (groesse von 2 Walnuessen)
\item[] 4 EL Öl (Sesamöl)
\item[] 6 EL Sojasauce
\item[] 1 TL Pfeffer oder Sichuanpfeffer
\item[] 2 TL Salz
\end{itemize}
Sosse:
\begin{itemize}
\item[] Dunkler Reisessig
\item[] Sojasosse
\item[] getrocknete Chili
\item[] Sesamoel
\end{itemize}

\subsubsection*{Directions}
\begin{enumerate}
\item Teig kneten und eine Stunde unter feuchtem Tuch ruhen lassen.
\item Chinakohl klein hacken und 2TL Salz dazugeben. Nach 30 Minuten in Kuechentuch geben und gut auspressen. 
\item Chinakohl mit restlichen Zutaten fuer die Fuellung mischen.
\item Teig zu Scheiben (Durchmesser=8cm) ausrollen und mit Fuellung zu Jiaozi verarbeiten.
\item Jiaozi bei ca 90 Grad 10 Minuten kochen.
\end{enumerate}
\pagebreak


\subsection{Eggplant in Garlic Sauce}
\subsubsection*{Ingredients (for 4 persons)}
Main things:
\begin{itemize}
  \item[] 1 lb eggplant – Japanese, Chinese, all good.
  \item[] 2 tsp vinegar – Chinkiang is nice, but rice wine or cider will substitute nicely.
\end{itemize}
Sauce:
\begin{itemize}
  \item[] 3 cloves garlic, smacked, then cut into small chunks
  \item[] 1 T ginger root, peeled and minced (=21 g)
  \item[] 2 tsb hot chili sauce, Sichuan if you can find it or Sriracha
  \item[] 2 T soy sauce – if you have light and dark, use 1 of each.
  \item[] 1 T brown sugar
  \item[] 1/3 c broth (low-sodium if canned) or water+1/2 tsp salt
\end{itemize}
Thickener:
\begin{itemize}
\item[] 2 tsp cornstarch or tapioca flour
\item[] 2 T water
\end{itemize}
Garnish:
\begin{itemize}
\item[] 1 scallion, minced
\item[] 1 tsp sesame oil
\end{itemize}
\subsubsection*{Directions}
\begin{enumerate}
%\item Cut eggplant into 2 inch cubes or into 1 inch thick rounds.
%\item Place the eggplant in a heatproof dish and steam until tender, approximately 10-20 m, – a knife should easily pierce it but not mush.
\item Aubergine in Auberginenbreitex8x20 mm dicke Scheiben schneiden.
\item Auberginenstuecke mit 2 T Oel bei mittlerer Hitze in die Pfanne geben und braten bis sie braeunlich werden. Ruehren noetig.
\item Combine the thickener ingredients and set aside.
\item Combine the sauce ingredients in a wok and bring to a rolling boil, reduce to a simmer and cook for 5 m, uncovered.
\item Add the eggplant and vinegar and stir gently from the bottom to coat with the sauce.
\item Heat through, then add the thickener, still stirring very gently. Cook 1-2 m to get rid of the starchy taste.
\item Slide onto a serving plate and garnish with scallion and sesame oil.
\end{enumerate}
\pagebreak

\subsection{Cucumber Salad (Beijing style)}
\subsubsection*{Ingredients (2 persons)}
\begin{itemize}
\item[] 1 	 Gurke(n)
\item[] 1/2 TL	 Salz
\item[]  etwas	 Pfeffer
\item[] 1 TL	 Zucker
\item[] 2 EL	 Essig (Weißwein- oder Kräuteressig, kein Apfelessig)
\item[] 2 Zehe/n	 Knoblauch, gepresst
\item[] 1/2 TL	 Chilisauce
\item[] 1 EL	 Öl (Sesamöl)
\end{itemize}
\subsubsection*{Directions}
\begin{enumerate}
\item Gurke waschen, viertel und die Kernen entfernen. Gurke in kleinen Würfeln schneiden (ca. 1,5-cm-dick). Mit Salz kurz kneten, 15 Minuten ziehen lassen. Dann mit restlichen Zutaten mischen und noch mal 15-20 Minuten ziehen lassen.
Den Salat kann man ein paar Tage in Kühlschrank aufbewahren.
\end{enumerate}
\pagebreak


\subsection{Steamed Fish in Black Bean Sauce}
Adapted non-steamed version
\subsubsection*{Ingredients (2 persons)}
\begin{itemize}
\item[] 2 thick firm white fish fillets or 1 1/2 lbs red snapper 
\item[] 2 tablespoons fermented black beans
\item[] 2 green onions, shredded into 1 1/2 inch long pieces OR 1 onion + 2 cloves of garlic OR 3-4 shallots
\item[] 4 slices ginger, shredded (=21 g)
\item[] 1 teaspoon sugar
\item[] 2 tablespoons sherry wine or 2 tablespoons rice wine
\item[] 1 tablespoon light soy sauce
\item[] 1 tablespoon peanut oil
\item[] cilantro (to garnish)
\end{itemize}
\subsubsection*{Directions}
\begin{enumerate}
\item Heat oil in pan and stew fish until fully cooked. Add some water if required. Remove from pan.
\item Food-prcoess onion, garlic and ginger and fry in pan until fragrant.
\item Add all liquid ingredients, black beans and sugar. Let boil on high heat shortly.
\item Add fish and water if needed.
\item Serve with cilantro.
\end{enumerate}
\pagebreak


\subsection{Stir Fry Bok Choi Sichuan Style}
\subsubsection*{Ingredients (4 persons)}
\begin{itemize}
\item[] 700 g	Pok choi
\item[] 2 T rice wine - ggf weniger bzw weglassen
\item[] 3-4  cloves of garlic 
\item[] 1 T ginger (=21 g)
\item[] 1 T sesame oil Sesamöl (dunkel) - ggf weglassen
\item[] 1 T Sichuan pepper
\item[] 3/4 T salt
\item[] 3 T oil
\end{itemize}
\subsubsection*{Directions}
\begin{enumerate}
\item Option: Fry sichuan pepper (1 T) in oil until truning brown. Remove sichuan pepper with spatula.
\item If big pok choi: Fry white part first until color changes
\item Add garlic, ginger, sichuan pepper. Fry until garlic changes color
\item Add green part of bok choi, salt and sesame oil. Stir. Remove soon after
\end{enumerate}
\pagebreak

%
%\subsection{Cabbage and Glass Noodle Stir-Fry}
%Mengenangaben muessen noch ueberarbeitet werden
%\subsubsection*{Ingredients}
%\begin{itemize}
%\item[] "small" package of glass noodles
%\item[] 1/2 of a small head of cabbage, cut into 7mm strips
%\item[] Dofu (optional)
%\item[] 1-2 carrots (optional)
%\item[] Salt
%\item[] 1/2 teaspoon sesame oil, plus an extra splash
%\item[] 1 tablespoon shaoxing wine, plus an extra splash
%\item[] dried chilis (optional)
%\item[] 2 Cloves of Garlic, finelly chopped
%\item[] 1 Scallion, sliced
%\item[] 1/4 teaspoon white pepper
%\item[] 1/2 teaspoon soy sauce
%\end{itemize}
%\subsubsection*{Directions}
%\begin{enumerate}
%\item Soak glass noodles.
%\item Fry scallions with garlic and chili.
%\item Add Ingredients and stirfry 
%\end{enumerate}
%\pagebreak
\section{Vietnam}

\subsection{Pho}
\subsubsection*{Ingredients (for 6-8 persons)}
Broth
\begin{itemize}
\item[] 2 onions, halved
\item[] 4" nub of ginger, halved lengthwise
\item[] 5-6 lbs of good beef bones, preferably leg and knuckle
\item[] 1 lb of beef meat - chuck, brisket, rump, cut into large slices [optional]
\item[] 6 quarts of water
\item[] 1 package of Pho Spices [1 cinnamon stick, 1 tbl coriander seeds, 1 tbl fennel seeds, 5 whole star anise, 1 cardamom pod, 6 whole cloves - in mesh bag]
\item[] 1 1/2 tablespoons kosher salt (halve if using regular table salt)
\item[] 1/4 cup fish sauce
\item[] 1 inch chunk of yellow rock sugar (about 1 oz) - or 1oz of regular sugar
\end{itemize}
Bowls
\begin{itemize}
\item[] 2 lbs rice noodles (dried or fresh)
\item[] cooked beef from the broth
\item[] 1/2 lb flank, london broil, sirloin or eye of round, sliced as thin as possible.
\item[] big handful of each: mint, cilantro, basil
\item[] 2 limes, cut into wedges
\item[] 2-3 chili peppers, sliced
\item[] 2 big handfuls of bean sprouts
\item[] Hoisin sauce
\item[] Sriracha hot sauce
\end{itemize}
\subsubsection*{Directions}
\begin{enumerate}
\item Char: Turn your broiler on high and move rack to the highest spot. Place ginger and onions on baking sheet. Brush just a bit of cooking oil on the cut side of each. Broil on high until ginger and onions begin to char. Turn over and continue to char. This should take a total of 10-15 minutes.

\item Parboil the bones: Fill large pot (12-qt capacity) with cool water. Boil water, and then add the bones, keeping the heat on high. Boil vigorously for 10 minutes. Drain, rinse the bones and rinse out the pot. Refill pot with bones and 6 qts of cool water. Bring to boil over high heat and lower to simmer. Using a ladle or a fine mesh strainer, remove any scum that rises to the top.

\item Boil broth: Add ginger, onion, spice packet, beef, sugar, fish sauce, salt and simmer uncovered for 1 1/2 hours. Remove the beef meat and set aside (you'll be eating this meat later in the bowls) Continue simmering for another 1 1/2 hours. Strain broth and return the broth to the pot. Taste broth and adjust seasoning - this is a crucial step. If the broth's flavor doesn't quite shine yet, add 2 teaspoons more of fish sauce, large pinch of salt and a small nugget of rock sugar (or 1 teaspoon of regular sugar). Keep doing this until the broth tastes perfect. Note: Wenn man fettige Knochen auskocht (z.B. Mark), dann wird die Suppe zu fettig. Fett abzuschoepfen ist einfach wenn man die Suppe klat werden laesst.

\item Prepare noodles and meat: Slice your flank/london broil/sirloin as thin as possible - try freezing for 15 minutes prior to slicing to make it easier. Remember the cooked beef meat that was part of your broth? Cut or shred the meat and set aside. Arrange all other ingredients on a platter for the table. Your guests will "assemble" their own bowls. Follow the directions on your package of noodles - there are many different sizes and widths of rice noodles, so make sure you read the directions. For some fresh rice noodles, just a quick 5 second blanch in hot water is all that's needed. The package that I purchased (above) - needed about 45 seconds in boiling water.

\item Ladling: Bring your broth back to a boil. Line up your soup bowls next to the stove. Fill each bowl with rice noodles, shredded cooked beef and raw meat slices. As soon as the broth comes back to a boil, ladle into each bowl. the hot broth will cook your raw beef slices. Serve immediately. Guests can garnish their own bowls as they wish.
\end{enumerate}
\pagebreak
\section{Thailand}

\subsection{Red Curry with Beef and Eggplant}
Am besten schmeckt es mit den Thai Auberginen, als alternative kann man grüne Bohnen reintun (aber nur wenns sein muss :)). 
Wichtig ist, dass man gute rote Curry Paste benutzt z.B. von Mae Ploy.
\subsubsection*{Ingredients (for 4 persons)}
\begin{itemize}
\item[] Öl, zum Braten
\item[] 2 EL	 Currypaste, rot, oder weniger
\item[] 600 g	 Rindfleisch, z.B. Rumpsteak
\item[] 2 EL Kokosmilchpulver (oder Kokosmilch) oder Wasser
\item[] 6 	 Kaffir-Limettenblätter
\item[] 100 g	 Auberginen (Maküa Püang), Thai-Erbsenauberginen
\item[] 4 	 Auberginen (Maküa Po), golfballgroß, grün oder grünweiß
\item[] 100g  grüne Bohnen (nicht in Originalrezept)
\item[] 3 EL  Fischsauce
\item[] 2  EL	 Limettensaft
\item[] 1 TL (=1/3 EL)	 Zucker, braun
\item[] 6 Stück	 Chilischote(n), rot, fingerlang (optional)
\item[] 2 Handvoll	 Thaibasilikum (Bai Horapa) oder Korinader (optional) 
\end{itemize}
\subsubsection*{Directions}
\begin{enumerate}
\item Fry meat and remove
\item Add curry paste, fry one minute
\item Add cocunutmilk/water (approx. half a cup) and kaffir lime leaves and meat (cook 15 minutes)
\item Add rest (cook 10 minutes)
\end{enumerate}
\subsubsection*{Comments}

\subsection{Basil Beef}
\subsubsection*{Ingredients (for 2 persons)}
\begin{itemize}
\item[] 2 tablespoons oil
\item[] 1 tablespoon garlic, minced
\item[] 1/2 lb beef, thinly sliced
\item[] 1 tablespoon chile, chopped (2 scharfe ohne kerne sind gut)
\item[] 1/4 cup onion, sliced
\item[] 1/4 carrot, julienned (optional)
\item[] 1/4 cup red bell pepper, sliced
\item[] 1 cup fresh basil leaf
\item[] 1 tablespoon oyster sauce
\item[] 2 tablespoons fish sauce
\item[] 1/2 tablespoon sugar
\item[] 1/8 teaspoon white pepper
\item[] cilantro, chopped for garnish
\end{itemize}
\subsubsection*{Directions}
\begin{enumerate}
\item Heat wok with oil, add garlic and chiles.
\item Add beef and stir fry for 1 minute.
\item Add carrots, onions and basil, stir for 1 minute.
\item Add remaining sauce: oyster sauce, fish sauce, sugar, and white pepper.
\item Stir fry until cooked, garnish with cilantro.
\item Server immediately with steamed rice.
\end{enumerate}
\pagebreak


\subsection{Vegetables in Oyster Sauce}
Eignet sich besonders als Zusatzgericht. Man kann dieses Gerich auch mit anderen asiatischen gruenen Gemuesen machen. 
\subsubsection*{Ingredients (for 2-3 persons)}
\begin{itemize}
\item[] Vegetable oil
\item[] 4-6 pak choi (or similar green vegetable)
\item[] 1 big onions, quaters (oder eine 1/2 gemuesezwiebel, achteln) auseinandergeblaettert
\item[] 3 Tbs (no MSG, z.B. Maekrua brand) oyster sauce 
\item[] 2 Tbs water
\item[] 1-2 tsp tablespoons fish sauce 
\item[] 1 tsp sugar
\item[] 3 clovevs garlic
\end{itemize}
\subsubsection*{Directions}
\begin{enumerate}
\item Mix sauce ingredients (oyster sauce, water, fish sauce, sugar)
\item Heat oil.
\item Add garlic, stir for 1 minute.
\item Add onions, stir for 1 minute.
\item Add vegetables
\item Add sauce
\item Fry for 3-5 minutes.
\end{enumerate}
\pagebreak


\subsection{Fish with Tamarind Sauce}
Eignet sich besonders als Zusatzgericht. Man kann dieses Gerich auch mit anderen asiatischen gruenen Gemuesen machen. 
\subsubsection*{Ingredients (for 2-3 persons)}
\begin{itemize}
\item[] Vegetable oil
\item[] 1 Fish (eigentlich ein ganzer, filet geht auch)
\item[] 3 sprigs cilantro 
\item[]  2 tablespoons fish sauce
\item[]  2 cloves garlic
\item[]  1 shallots
\item[]  2 1/2 tablespoons sugar
\item[]  1 tablespoon tamarind paste
\end{itemize}
\subsubsection*{Directions}
\begin{enumerate}
\item Fisch braten. Ganzer Fisch: deep fry wok (nie probiert)
\item Alles andere ausser dem Koriander im Food-processor zerkleinern. In eine Phanne geben und aufkochen lassen. Ggf noch etwas Wasser dazu.
\item Sauce auf dem Teller ueber den Fisch geben und mit Koriander garnieren.
\end{enumerate}
\pagebreak


\section{Austria}

\subsection{Gemüsesuppe à la Oma}
Anhand des Originalrezepts und weiteren Erlaeuterungen von der Oma. Es koennen alle moeglichen Gemuese benutzt werden, die Zutatenliste ist nur ein Vorschlag. Kohlrabi sind wichtig, koennen im Winter durch Knollenselerie ersetzt werden. Wichtig ist natuerlich auch die richtige Menge Salz. Die Mengenangabe 0.35-0.4 EL(Tbs.)/Liter bezieht sich auf das gesammte Suppenvolumen, welches man von der Topfgroesse abschaetzen kann. Das Rezept unten gibt ca. 4-5 Liter.
\subsubsection*{Ingredients}
\begin{itemize}
\item[] 2 Kohlrabi
\item[] 1/2-1 Blumenkohl oder1 Brocoli
\item[] 3-4 Tomaten
\item[] 5 Kartoffel
\item[] 1/4 kg Erbsen optional
\item[] 3 Karotten
%\item[] 1 Teelöffel Kümmel
\item[] Pfeffer, ganz und gemahlen
\item[] Knoblauch
\item[] Salz (0.35-0.4 EL(Tbs.)/Liter Suppenvolumen)
\item[] frischer Dill
\item[] frischer Petersil
\end{itemize}
Einbrenn
\begin{itemize}
\item[] 1 Zwiebel
\item[] Öl
\item[] 2 Esslöffel Mehl
\end{itemize}
\subsubsection*{Directions}
\begin{enumerate}
\item Einbrenn: Öl in einem Topf heiß machen und die Zwiebeln dazu gegeben. Bevor die Zwiebeln 
braun werden Mehl dazu geben. Warten bis das Mehl leicht braeunlich wir und Wasser aufgiessen.
\item Mehr Wasser (ca. 2 Liter) und Gewuerze dazu geben. Dann Gemuese (recht klein geschnitten) in Reihenfolge der Kochdauer zugeben
\end{enumerate}
\pagebreak


\subsection{Geselchtes au Peter}
\subsubsection*{Ingredients}
\subsubsection*{Directions}
\begin{enumerate}
\item Liebstoeckl, Mayoran, Pfeffer, Thymian Kuemmel in heisses Wasser gegen und 3 Tage stehen lassen.
\item Fleisch in Salz Pfeffer Majoran und Knoblauch einreiben und in Behaeltnisgeben. Mit Brettern beschweren und 2-3 Tage liegen lassen
\item Gewuerzwasser ueber das Fleisch geben und 14 Tage stehen lassen.
\item Fleisch abschwemmen und Fleisch mit Spicknadel und Schnur aufhaengen.
\item 1.5 Tage mit Buchenholz raeuchern.
\end{enumerate}
\pagebreak


\subsection{Geselchtes a la Martha}
\subsubsection*{Ingredients}
\subsubsection*{Directions}
\begin{enumerate}
\item ca. 2 dag Pökelsalz auf 1 kg Rohfleisch
\item mit  Kümmel Knoblauch Lorbeerblättern - eine Mischung bereiten, und damit das Fleisch einreiben
\item Die eingeriebenen Fleischstücke in ein enges Plastikwanndl  legen und bei ca. 8Grad 
\item am Besten Fleischstücke auf 2 Lagen zB übereinander legen
\item nach ca. 3-4 Tagen - umschichten des Fleisches (das obrig liegende nach unten - und das unten liegende oben auf legen)
\item nach insgesamt 8-10 Tagen Beiz-Ruhedauer kann dann geselcht werden. 
\item Selchen bei uns in unserer Selch ca. 1 Tag - mit Buchenholz, Wacholderästen, ... aber ist jede Selch anders zum handhaben...
\end{enumerate}
\pagebreak

\section{Bavaria}

\subsection{Apfelkuchen vom Blech mit Streusel}
Mit genau der richtigen Menge Zucker - etwas weniger als sonst.
\subsubsection*{Ingredients (fuer ein Blech)}
Teig
\begin{itemize}
\item[] Packung Vaniellezucker
\item[] 155 g Mehl
\item[] 100 g Butter
\item[] 45 g Zucker
\end{itemize}
Aepfel
\begin{itemize}
\item[] 1 kg Saeuerliche Aepfel
\end{itemize}
Streusel
\begin{itemize}
\item[] 100 g Butter
\item[] 57 g Zucker
\item[] 143 g Mehl
\end{itemize}
\subsubsection*{Directions}
\begin{enumerate}
\item Teig kneten und auf gefettetem Blech oder Backpapier verteilen.
\item Aepfel in schmale Spalten schneiden und auf den Teig geben
\item Streusel auf den Aepfeln verteilen
\item bei 175 Grad (vorgeheizt) backen
\end{enumerate}
\pagebreak

\subsection{Schweinsbraten}
Schweinebraten in zwei Varianten. Variante1: Bayrisch mit Biersosse. Variante 2: (a la Mama) mit Nelken
\subsubsection*{Ingredients (3-4 Persons)}
\begin{itemize}
\item[] ca. 1kg Schweinbraten z.B. Schulter mit Schwarte
\item[] 1.5 kg Kartoffeln
\item[] 400 g Karotten
\item[] 400 g Zwiebeln
\item[] 5 Knoblauchzehen (klein geschnitten und optional teilweise in "Zaehne" geschnitten )
\item[] Kuemmel 
\item[] 5 Gewuerznelken (Variante 2)
\item[] Pfeffer (Variante 2)
\item[] Salz (1tsp/400g Fleisch) + 0.5 tsp auf Gemuese - ein klein wenig mehr kanns auch sein
\item[] Pfeffer
\item[] Bier
\end{itemize}
\subsubsection*{Directions}
\begin{enumerate}
\item Schweinebraten kraeftig salzen und mit Knoblauch und Oel einreiben, optional Schwarte mit Knoblauch spicken. Bei Variante 2 Schwarte mit Nelken spicken und Braten pfeffern. Danach mit Schwarte nach unten in einen gusseisenen Topf oder Roemertopf mit Deckel geben. Bei 210 grad eine Stunde in den Backofen geben.
\item Kartoffeln, Karotten und Zwiebel schaelen und in ca. 3 cm grosse Stuecke schneiden.
\item Topf aus dem Ofen holen, Braten wenden, Gemuese dazugeben. Grosszuegig Kuemmel drauf geben und nochmal etwas Oel und Salz drauf. Danach nochmal eine Stunde bei 190 Grad in den Ofen.
\item Deckel entfernen,Temperatur auf 210 Grad erhoehen und alle 10 Minuten etwas Bier (in Variante 2 Wasser) mit Loeffel ueber die Schwarte geben. Am Ende sollte die Kruste braun und knusprig sein und die Kartoffeln oben auch etwas braeunlich. Diese letzte Phase kann auch nochmal eine Stunde dauern. Je nach geometrie muss man die Kartoffeln vor dem Braten herausnehmen.
\end{enumerate}
\pagebreak


\section{Italy}
\subsection{Pasta Arrabiata}
\subsubsection*{Ingredients (for 2 persons)}
\begin{itemize}
\item[] 1 Dose ganze geschälte Tomaten 
\item[] 3 korrekte Knoblauchzehen 
\item[] 1 scharfe große Chili, je nach Geschmack 
\item[] Olivenöl 
\item[] Salz
\item[] Nudeln
\end{itemize}
\subsubsection*{Directions}
\begin{enumerate}
\item Im Olivenöl Knoblauch und Chili anbraten, die Tomaten dazu, salzen und mindestens eine halbe Stunden köcheln lassen. Dann die Tomaten zerdrücken und kurz weiterköcheln lassen. Z.B. während man die Nudeln aufsetzt... 
\end{enumerate}
\pagebreak

\subsection{Fischsuppe}
Dieses Gericht ist eigentlich nicht italienisch, sondern eine Variation von der Gemuesesuppe. Gewuerze sind eher mediterran.
\subsubsection*{Ingredients}
\begin{itemize}
\item[] Zutaten wie Gemüsesuppe aber ohne Erbsen, Dill, Petersil
\item[] Fisch (Meeresfisch oder auch Pangasius)
\item[] Thymian
\item[] Zitrone
\end{itemize}
\subsubsection*{Directions}
\begin{enumerate}
\item Zubereitung wie Gemuesesuppe
\item 15 Minuten vor dem servieren den Fisch (ggf. noch gefroren) dazugeben
\item Je nach Säuregehalt (Tomate) noch mit Zitrone abschmecken
\end{enumerate}
\pagebreak

\section{India}
\subsection{Dal Makhani}
Dal mit braunen und grünen Linsen (Dal Makhani) Für 4 Personen 
\subsubsection*{Ingredients}
\begin{itemize}
\item[] 60g grüne Linsen 
\item[] 60g braune Linsen 
\item[] 1 Zwiebeln, fein gehackt 
\item[] 2 Zehen Knoblauch, fein gehackt 
\item[] 2 Esslöffel Öl 
\item[] 1/2 TL Haldi Pulver 
\item[] 1/2 TL Chili Pulver 
\item[] 1/2 TL Ingwer Pulver oder frischer Ingwer 
\item[] 1/2 TL Currypulver 
\item[] 1/2 TL Garam Masala 
\item[] 1 TL Salz 
\item[] 3 Tomaten gehackt (Alternativ 1 kl. Büchse gehackte Tomaten) 
\item[] 3 TL Butter 
\item[] 1dl Rahm 
\item[] Zitronensaft 
\end{itemize}
\subsubsection*{Directions}
\begin{enumerate}
\item Linsen gut waschen und abtropfen lassen
\item 400ml Wasser aufkochen, Linsen, Haldi, Chili und Salz bei kleiner Hitze 30min weichkochen. 
\item Zwiebeln und Knoblauch im Öl erhitzen und glasig rösten. Gehackte Tomaten dazugeben, mit dem Ingwer und Curry einkochen lassen. Diese Tomatenpaste zum Dal geben, ev. Wasser dazu giessen und 20min. weitergaren lassen. 
\item Mit Butter und etwas Rahm verfeinern. 
\item Nach 5 Minuten Garam Masala und den restlichen Rahm dazugeben, 10 Minuten weiterköcheln. 
\item Am Schluss nach Bedarf etwas Zitronensaft dazugeben. Frischer Koriander eignet sich ebenfalls als Zugabe.
\end{enumerate}
\pagebreak

\section{Singapore, Malaysia}
\subsection{Chicken Rice}
\subsubsection*{Ingredients}
Chicken:
\begin{itemize}
\item[] 
\item[] kosher salt
\item[] 4'' section of fresh ginger, in 1/4'' slices
\item[] 2 stalks green onions, cut into 1" sections (both the green and white parts)
\item[] 1 teaspoon sesame oil
\end{itemize}
Rice:
\begin{itemize}
\item[] 2 tablespoon chicken fat or 2 tbsp vegetable oil
\item[] 3 cloves garlic, finely minced
\item[] 1'' section of ginger, finely minced
\item[] 2 cups long-grain uncooked rice, washed and soaked in cool water for 10 min or longer
\item[] 2 cups reserved chicken poaching broth
\item[] 1/2 teaspoon sesame oil
\item[] 1 teaspoon kosher salt
\end{itemize}
Sauce à la Jelly:
\begin{itemize}
\item[] 2-3 ganze Stange thailändischer Koriander (einschließlich Wurzel), gehackt
\item[] 3-4 Zehen Knoblauch, fein gehackt
\item[] 3-4 thailändische Chilli Schotten, fein gehackt
\item[] 1 mittlegroße Knolle Ingwer, geschält und fein gehackt
\item[] ca. 2 Kochlöffel Hühnerbrühe
\item[] 2-3 EL thailändische Sojabohnenpaste (lieber von "Healthy Boy Brand")
\item[] 2 TL Zucker
\item[] ca. 1 TL Fischsoße
\item[] frisch gepresster Limettensaft aus ca. 1 Limette
\end{itemize}

\subsubsection*{Directions}
Chicken and Rice:
\begin{enumerate}
\item To clean the chicken, with a small handful of kosher salt, rub the chicken all over, getting rid of any loose skin and dirt. Rinse chicken well, inside and outside. Season generously with salt inside and outside. Stuff the chicken with the ginger slices and the green onion. Place the chicken in a large stockpot and fill with cold water to cover by 1 inch. Bring the pot to a boil over high heat, then immediately turn the heat to low to keep a simmer. Cook for about 30 minutes more (less if you're using a smaller chicken). Check for doneness by sticking a chopstick into the flesh under the leg and see if the juices run clear or insert a thermometer into the thickest part of the thigh not touching bone. It should read 170F.

\item When the chicken is cooked through, turn off the heat and remove the pot from the burner. Immediately lift and transfer the chicken into a bath of ice water to cool and discard the ginger and green onion. Don't forget to reserve the poaching broth for your rice, your sauce, and the accompanying soup. The quick cooling will stop the cooking process, keeping the meat soft and tender, and giving the skin a lovely firm texture.

\item Rice: In a sauce pan heat 2 tablespoons of cooking oil over medium-high heat. When hot, add the ginger and the garlic and fry until your kitchen smells like heaven. Be careful not to burn the aromatics! Add in your drained rice and stir to coat, cook for 2 minutes. Add the sesame oil, mix well. In the same sauce pan, add 2 cups of broth, add salt and bring to a boil. Turn the heat down to low, cover the pot and cook for 15 minutes. Remove from heat and let sit (with lid still on) for 5-10 minutes more.

\item While your rice is cooking, remove the chicken from the ice bath and rub the outside of the chicken with the sesame oil. Carve the chicken.

\item Blend your chili sauce ingredients in a blender until smooth.

\item Soup: Just before serving, heat up the broth, taste and season with salt as necessary.

\item Serve the chicken rice with chili sauce, dark soy sauce, cucumber slices, and a bowl of hot broth garnished with cilantro or scallions
\end{enumerate}
Sauce à la Jelly:
\begin{enumerate}
\item Sojabohnenpaste in Hühnerbrühe in einem Topf bei schwacher Hitze verdünnen und erwärmen
\item Zucker und Fischsoße geben, rühren
\item Koriander, Knoblauch, Chilli und Ingwer unterrühren
\item Limittensaft dazu geben, bei stärkerer Hitze kurz aber gut unterrühren, dann den Herd ausschalten
\end{enumerate}
\pagebreak

\subsection{Pisang Goreng}
For more crispyness some peopole replace some of the rice flour with cornstarch, also put batter in fridge before frying, fry hot (185C)
\subsubsection*{Ingredients}
\begin{itemize}
\item[]     4 bananas
\item[]     75 gram rice flour
\item[]     1 tablespoon tapioca flour
\item[]     1/4 teaspoon salt
\item[]     125 ml water
\item[]     oil for deep frying
\end{itemize}
\subsubsection*{Directions}
\begin{enumerate}
\item     Combine rice flour, tapioca flour, salt, and water in a mixing bowl.
\item     Peel bananas and cut each into two. Make three slices along its length, but keep the bottom 1 inch intact, so it can be opened up like a fan.
\item     Heat enough oil in a pot for deep frying.
\item     Dip the bananas into the batter and fry until golden brown and crispy, about 3-4 minutes. Remove and drain on paper towel or wire rack. Serve immediately.

\end{enumerate}
\pagebreak

