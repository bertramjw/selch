\section{Originalrezepte
}
\subsection{Geselchtes au Peter}
\subsubsection*{Ingredients}
\subsubsection*{Directions}
\begin{enumerate}
\item Liebstoeckl, Majoran, Pfeffer, Thymian Kuemmel in heisses Wasser gegen und 3 Tage stehen lassen.
\item Fleisch in Salz Pfeffer Majoran und Knoblauch einreiben und in Behaeltnisgeben. Mit Brettern beschweren und 2-3 Tage liegen lassen
\item Gewuerzwasser ueber das Fleisch geben und 14 Tage stehen lassen.
\item Fleisch abschwemmen und Fleisch mit Spicknadel und Schnur aufhaengen.
\item 1.5 Tage mit Buchenholz raeuchern.
\end{enumerate}
Anmerkung: Die Lisi macht es auch so, aber mit Salz als einizges Gewuerz
\pagebreak


\subsection{Geselchtes a la Martha}
\subsubsection*{Ingredients}
\subsubsection*{Directions}
\begin{enumerate}
\item ca. 2 dag Pökelsalz auf 1 kg Rohfleisch
\item mit  Kümmel Knoblauch Lorbeerblättern - eine Mischung bereiten, und damit das Fleisch einreiben
\item Die eingeriebenen Fleischstücke in ein enges Plastikwanndl  legen und bei ca. 8Grad 
\item am Besten Fleischstücke auf 2 Lagen zB übereinander legen
\item nach ca. 3-4 Tagen - umschichten des Fleisches (das obrig liegende nach unten - und das unten liegende oben auf legen)
\item nach insgesamt 8-10 Tagen Beiz-Ruhedauer kann dann geselcht werden. 
\item Selchen bei uns in unserer Selch ca. 1 Tag - mit Buchenholz, Wacholderästen, ... aber ist jede Selch anders zum handhaben...
\end{enumerate}
Anmerkung: Sie benutzt aber normales Salz statt Poekelsalz
\pagebreak

\section{Versuche}
\subsection{Nr. 1 - Dezember 2015}
\subsubsection*{Fleisch}
von Kurt Brunner:
\begin{enumerate}
\item Schweinsbrustschnitte 0.620 g
\item Schweinsbraten (Laffe) 1.142 g
\end{enumerate}
%
\subsubsection*{Rezept/Verarbeitung}
Fleisch eingerieben mit
\begin{enumerate}
\item Salz (1 TL pro 400 g Fleisch)
\item Majoran (2 EL)
\item Pfeffer (1 EL)
\item Knoblauch (2 Zehen)
\item Kümmel 	(1 EL)
\end{enumerate}
danach 3 Tage im Kühlschrank und danach mit der ebenfals 3 Tage alten Lösung zusammen geführt.
\begin{enumerate}
\item Heisses Wasser (2.2 l)
\item Majoran (2 EL)
\item Pfeffer (1 EL)
\item Thymian (2 EL)
\item Kümmel 	(1 EL)
\end{enumerate}
Wie lange wir das Fleisch nach dem Zusammenführen im Kühlschrank hatten, weiss ich leider nicht mehr genau ca. 14 Tage.
Vor dem Räuchern ist das Fleisch dann noch einen Tag (oder mehr?) abgehangen draussen vor dem Keller im Topf ohne Lösung.

\subsubsection*{Räuchern}
Weiss ich leider nicht mehr so genau.
Aber ziemlich genau einen Sack voll Buchenholzspäne in 4 Tagen.
Mit der Vorrichtung vom Tontopf ca. 4 Stunden Räuchern möglich. Mit Papier oder Karton Rolle basteln, damit Späne nicht rausfallen beim Befüllen. Sobald Tontopf positioniert ist, Papier rausziehen, Anzündhilfe aus Stroh darunter legen und Feuer im Kamin (Loch in den Spänen) aufsteigen lassen. Wenn alles schön glüht, Anzündwürfel raus und Deckel drauf.


